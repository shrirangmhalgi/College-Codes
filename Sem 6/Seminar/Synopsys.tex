% Generated by GrindEQ Word-to-LaTeX 
\documentclass{article} %%% use \documentstyle for old LaTeX compilers

\usepackage[english]{babel} %%% 'french', 'german', 'spanish', 'danish', etc.
\usepackage{amssymb}
\usepackage{amsmath}
\usepackage{txfonts}
\usepackage{mathdots}
\usepackage[classicReIm]{kpfonts}
\usepackage[dvips]{graphicx} %%% use 'pdftex' instead of 'dvips' for PDF output

% You can include more LaTeX packages here 


\begin{document}

%\selectlanguage{english} %%% remove comment delimiter ('%') and select language if required


\noindent \textbf{}

\noindent \textbf{\eject }

\noindent \textbf{SYNOPSYS.}

\noindent \textbf{}

\noindent \textbf{Name of the Student:} Shrirang Rajendra Mhalgi.

\noindent \textbf{Roll No:} 322008

\noindent \textbf{Branch:} Computer Science

\noindent \textbf{Email ID:}~shrirangmhalgi@gmail.com

\noindent \textbf{Mobile:} +91-9767916351

\noindent \textbf{Title of the topic:} Semantic and Sentimental Analysis using NLP.

\noindent \textbf{Area of topic:} Semantic and Sentimental Analysis using NLP.

\noindent 

\noindent 

\noindent \textbf{Abstract : }

With the increase in data, there is a need to understand the correct and relevant meaning of the data and understand the underlying hidden sentiments which come out of it. Semantic word spaces have been very useful and can express the meaning of longer phrases in a principled way. Further progress towards understanding compositionality in tasks such as sentiment detection requires richer supervised training and evaluation resources and more powerful models of composition. To capture the effects of negation and its scope at various tree levels for both positive and negative phrases is a hot and trending topic of the 21${}^{st}$ century. 

Natural Language Processing is a technique which can be used to do accurate analysis of data and its underlying hidden sentiments and is widely used. Although getting 100\% accuracy is next to impossible, but scientists have made a progress of achieving a fair amount of high accuracy. 

With the help of accurate data, it is much easier to do actual and accurate analysis of data which can be further used by the experts to derive much more concrete results.

\noindent \eject 



\noindent \textbf{Keywords : }

\textbf{ }Semantics, Meanings, Linguistic, Pragmatic, Sentiments, Emotions, Language Processing, Opinion Mining, Natural language processing.

\noindent \textbf{}

\noindent \textbf{Brief about contents :}

Human language is special for several purposes. It is specifically constructed to convey the speakers / writers meaning. The human brain is very difficult to understand. The capability of the languages that the brain can process is tremendous. In today's world, a normal person knows almost 3 to 4 languages which include English as the universal language, their regional native language and some extra languages. It is strange that not only that the person knows all the languages but also, he is able to understand the hidden sentiments behind it.

A perfect understanding of language by a computer would result in an AI that can process the whole information that is available on the internet, which in turn would probably result in artificial general intelligence.

\noindent \textbf{Semantic Analysis : }

Semantics is all about meaning. It cannot be just defined for language but also can be defined for many different things such as arithmetic expressions. 

\noindent \textbf{Types of Semantic Analysis : }

In semantic analysis we assign each word some sort of meaning and combine the meanings to form a phrase, sentence, paragraphs, stories, narration, etc.

\begin{enumerate}
\item  Entailment :
\end{enumerate}

  In entailment one fact follows from the other.

\noindent 

\noindent 

\noindent \eject 

\noindent 

\begin{enumerate}
\item  Presupposition :
\end{enumerate}

We may draw some conclusions based on our observations :

\begin{enumerate}
\item  Understanding meaning :
\end{enumerate}

\noindent If the agent hears the sentence and acts accordingly then it is said that he has understood the meaning.

\noindent \textbf{Sentimental Analysis : }

Sentiment Analysis also known as~\textit{Opinion Mining}~is a field within~Natural Language Processing~(NLP) that builds systems that try to identify and extract opinions within text. 

\noindent \textbf{Types of Sentimental Analysis :} 


\subparagraph{ Fine-grained Sentiment Analysis : Sometimes you may be also interested in being more precise about the level of polarity of the opinion such as positive negative or neutral. }


\subparagraph{ Emotion detection : Emotion detection aims at detecting emotions like, happiness, frustration, anger, sadness, and the like. }

\begin{enumerate}
\item  \textbf{Aspect-based Sentiment Analysis : }Usually, when analyzing the sentiment in subjects, for example products, you might be interested in not only whether people are talking with a positive, neutral, or negative polarity about the product, but also which particular aspects or features of the product people talk about. 
\end{enumerate}


\subparagraph{ Intent Analysis : Intent analysis basically detects what people want to do with a text rather than what people say with that text. Look at the following examples:}


\subparagraph{ Multilingual sentiment analysis : Multilingual sentiment analysis usually uses a lot of preprocessing is needed, and makes use of a number of resources. }

\noindent \textbf{\eject }

\noindent \textbf{References :}

[1.] https://en.wikipedia.org/wiki/Natural\_language

\noindent [2.]https://www.expertsystem.com/natural-language-process-semantic-analysis-definition/

[3.] https://simplicable.com/new/natural-language-processing

[4.] https://monkeylearn.com/sentiment-analysis/

[5.] https://www.paralleldots.com/sentiment-analysis

\noindent [6. ] S. Miller, H. Fox, L. Ramshaw, and R. Weischedel. A novel use of statistical parsing to extract information from text. Applied Natural Language Processing Conference (ANLP), 2000. 

\noindent [7.] S. Miller, J. Guinness, and A. Zamanian. Name tagging with word clusters and discriminative training. In Conference of the North American Chapter of the Association for Computational Linguistics \& Human Language Technologies (NAACL-HLT), pages 337--342, 2004. 

\noindent [8.] P. Koomen, V. Punyakanok, D. Roth, and W. Yih. Generalized inference with multiple semantic role labeling systems (shared task paper). In Conference on Computational Natural Language Learning (CoNLL), pages 181--184, 2005.

\noindent [9.] A Mnih and G. E. Hinton. Three new graphical models for statistical language modelling. In International Conference on Machine Learning (ICML), pages 641--648, 2007. 


\end{document}

